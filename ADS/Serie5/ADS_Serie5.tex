\documentclass[12pt]{article}

\usepackage{amsmath}
\usepackage{amssymb}    % noch mehr mathematische Symbole
\usepackage{german}
\usepackage[utf8]{inputenc}

\usepackage{fancyhdr}
\pagestyle{fancy}
\lhead{Gruppe David Gruner}
\rhead{Dimitrios D. Antoulis / Timm-O. Stooß}
\cfoot{\thepage}
\renewcommand{\headrulewidth}{0.4pt}
\renewcommand{\footrulewidth}{0pt}

\usepackage{geometry}                                                                    % Rand einstellen
\geometry{verbose,a4paper,tmargin=25mm,bmargin=25mm,lmargin=30mm,rmargin=35mm}

% Für natürliche Zahlen N, reelle Zahlen R und so weiter ...
\AtBeginDocument{%
  \let\mathbb\relax
  \DeclareMathAlphabet\PazoBB{U}{fplmbb}{m}{n}
  \newcommand{\mathbb}{\PazoBB}
}
\newcommand {\N}{\PazoBB{N}}   % natürliche Zahlen
\newcommand {\Z}{\PazoBB{Z}}   % ganze Zahlen
\newcommand {\Q}{\PazoBB{Q}}   % rationale Zahlen
\newcommand {\R}{\PazoBB{R}}   % reelle Zahlen


\newcommand {\Pm}{\mathcal{P}} % Potenzmenge
\newcommand {\Rm}{\mathcal{R}} % kalligravarphisches R
\newcommand {\Om}{\mathcal{O}} % kalligravarphisches O


\begin{document}

\subsection*{ADS - Serie 5}
\subsection*{Aufgabe 5.2}
\begin{enumerate}
\item[Vor.:] Seien $f, g, h, f_i: \N \to \R_{>0}, i\in\N$ gegeben.
\item[a) Beh.:] 
          $\big(f\in \Om(g) \land g \in \Om(f) \big) \Longleftrightarrow \big(f \in \Theta(g) \land g\in \Theta(f) \big)$
\item[Bew.:] '$\Longrightarrow$':
             Es gelte $f\in \Om(g)$ und $g \in \Om(f)$. \\
             Also gilt $\exists C_0 \in \R_{>0} : \exists n_0 \in \N : \forall n \in \N_{\geq n_0} : f(n) \leq C_0 \cdot g(n)$ \\
                   und  $\exists C_1 \in \R_{>0} : \exists n_1 \in \N : \forall n \in \N_{\geq n_1} : g(n) \leq C_1 \cdot f(n)$. \\
             Es seien $C_0, n_0, C_1$ und $n_1$ entsprechend gewählt. 

            Sei $C_2 = \frac{1}{ C_0}$.   Dann gilt für $n\geq$ $n_0$:
                      \begin{align*}
                                                         f(n)                       &  \leq C_0 \cdot g(n)              \\
                                          \frac{1}{ C_0}   \cdot f(n)  &  \leq g(n)                               \\
                                                            C_2   \cdot f(n)  &  \leq g(n)                               \\
                      \end{align*}
          Also gilt für max$(n_0, n_1)$:
                                     $g(n) \geq C_2 \cdot f(n)$ und $g(n) \leq C_1 f(n)$. Damit gilt $g \in \Theta (f)$.

          Sei $C_3 = \frac{1}{ C_1} $. Dann gilt für $n\geq$ $n_1$:
                      \begin{align*}
                                                        g(n)                                  &  \leq C_1 \cdot f(n)              \\
                                                     \frac{1}{ C_1}   \cdot f(n)  &  \leq f(n)                               \\
                                                                      C_3   \cdot f(n)  &  \leq f(n)                               \\
                      \end{align*}
                   Also gilt für $n\geq$ max$(n_0, n_1)$:
                                     $f(n) \geq C_3 \cdot g(n)$ und $f(n) \leq C_0 \cdot g(n)$. Damit gilt $g \in \Theta (f)$.\vspace{0.5cm}
                   
                       '$\Longleftarrow$' folgt direkt aus der Definition von $\Theta$.


\item[b) Beh.:] 
          $\big(f\in \Om(g) $ und $ h \in \Omega(g) \big) \Longrightarrow f \in \Om(h)$
\item[Bew.:] Es gelte $f\in \Om(g) \land h \in \Omega(g)$.  \\
                     Dann gilt 
                     $\exists C_0\in  \R_{>0} : \exists n_0 \in \N : \forall n \in \N_{\leq n_0} : f(n) \leq C_0 \cdot g(n)$ \\
                     und $\exists C_1\in \R_{>0} : \exists n_1 \in \N : \forall n \in \N_{\leq n_1} : h(n) \geq C_1 \cdot g(n)$ \\
                     Sei $C_2 = \frac{C_0}{C_1}$. Dann gilt für $n\geq$ max$(n_0, n_1)$:
                      \begin{align*}
                                                        f(n)                       &  \leq C_0 \cdot g(n)                                    \\
                                                        f(n)                       &  \leq C_0 \cdot \frac {C_1} {C_1} \cdot g(n)          \\
                                                                                     &  \leq \frac{C_0}{C_1} \cdot h(n)                           \\
                      \end{align*}
                      Also ist $f \in\Om(h)$.



\item[c) Beh.:] 
          $\big( \forall i\in\{1, \dots, k\} : f_i \in \Om(g) \big) \Longrightarrow \sum^k_{i=1} f_i \in\Om(g)$
\item[Bew.:] Induktion über $k$
\item[IA.:] Sei $k=1$ und es gelte $\forall i\in\{1, \dots, k\} : f_i \in \Om(g)$.\\
                  Dann ist  $\sum^k_{i=1} f_i = \sum^1_{i=1} f_i = f_1 \in \Om(g)$.
\item[IV.:] Sei $k\in \N$  und es gelte $\big( \forall i\in\{1, \dots, k\} : f_i \in \Om(g) \big) \Longrightarrow \sum^k_{i=1} f_i \in\Om(g)$.

\item[IS.:] zz.: $\big( \forall i\in\{1, \dots, k+1\} : f_i \in \Om(g) \big) \Longrightarrow \sum^{k+1}_{i=1} f_i \in\Om(g)$\\

                      Es gelte: $\forall i\in\{1, \dots, k+1\} : f_i \in \Om(g)$\\
                      Dann gilt insbes.: $\exists C_1 \in \N : \exists n_1\in \N : \forall n \in\N_{>n_1} : f_{k+1}(n)  \leq C_1 \cdot g(n)$ \\
                      und nach IV gilt: $\exists C_0 \in \N : \exists n_0\in \N : \forall n \in\N_{>n_0} :  \sum^k_{i=1} f_i(n)  \leq C_0\cdot g(n)$

                     Sei $C_2 = C_0 + C_1$. Dann gilt für $n>$ max$(n_0, n_1)$:
                      \begin{align*}
                                                                        &  \sum^k_{i=1} f_i (n)                               & \quad \leq  \quad& C_0\cdot g(n)                               \\  
                          \Longleftrightarrow  \quad    &  f_{k+1}(n)  + \sum^k_{i=1} f_i (n)          & \quad \leq  \quad &C_0\cdot g(n)   + f_{k+1}(n)                              \\
                          \Longleftrightarrow  \quad    &   \sum^{k+1}_{i=1} f_i (n)                    & \quad \leq  \quad& C_0\cdot g(n)   +       C_1 \cdot g(n)                         \\
                          \Longleftrightarrow  \quad    &   \sum^{k+1}_{i=1} f_i (n)               & \quad \leq  \quad& (C_0 +  C_1) \cdot g(n)                         \\
                          \Longleftrightarrow  \quad    &   \sum^{k+1}_{i=1} f_i (n)                & \quad \leq  \quad &C_2 \cdot g(n)                         \\
                      \end{align*}
                     Also ist $ \sum^k_{i=1} f_i \in\Om(g)$.


\item[d) Beh.:] 
          $\big( \forall i\in\{1, \dots, k\} : f_i \in \Omega(g) \big) \Longrightarrow \prod^k_{i=1} f_i \in\Omega(g)$
\item[Bew.:] Induktion über $k$
\item[IA.:] Sei $k=1$ und es gelte $\forall i\in\{1, \dots, k\} : f_i \in \Omega(g)$.\\
                  Dann ist  $\prod^k_{i=1} f_i = \prod^1_{i=1} f_i = f_1 \in \Omega(g)$.
\item[IV.:] Sei $k\in \N$  und es gelte $\big( \forall i\in\{1, \dots, k\} : f_i \in \Omega(g) \big) \Longrightarrow \prod^k_{i=1} f_i \in\Omega(g)$.

\item[IS.:] zz.: $\big( \forall i\in\{1, \dots, k+1\} : f_i \in \Omega(g) \big) \Longrightarrow \prod^{k+1}_{i=1} f_i \in\Omega(g)$\\

                      Es gelte: $\forall i\in\{1, \dots, k+1\} : f_i \in \Omega(g)$\\
                      Dann gilt insbes.: $\exists C_1 \in \N : \exists n_1\in \N : \forall n \in\N_{>n_1} : f_{k+1}(n)  \geq C_1 \cdot g(n)$ \\
                      und nach IV gilt: $\exists C_0 \in \N : \exists n_0\in \N : \forall n \in\N_{>n_0} :  \prod^k_{i=1} f_i(n)  \geq C_0\cdot g(n)$ 

                      Sei $C_2 = C_0 \cdot  C_1$. Dann gilt für $n>$ max$(n_0, n_1)$:
                      \begin{align*}
                                                                        &  \prod^k_{i=1} f_i (n)                                    & \quad \geq  \quad& C_0\cdot g(n)                               \\  
                          \Longleftrightarrow  \quad    &  f_{k+1}(n) \cdot \prod^k_{i=1} f_i (n)          & \quad \geq  \quad &C_0\cdot g(n)   \cdot  f_{k+1}(n)                              \\
                          \Longleftrightarrow  \quad    &   \prod^{k+1}_{i=1} f_i (n)                    & \quad \geq  \quad& C_0\cdot g(n)   \cdot       C_1 \cdot g(n)                         \\
                          \Longleftrightarrow  \quad    &   \prod^{k+1}_{i=1} f_i (n)               & \quad \geq  \quad& (C_0 \cdot  C_1) \cdot g(n)                         \\
                          \Longleftrightarrow  \quad    &   \prod^{k+1}_{i=1} f_i (n)                & \quad \geq  \quad &C_2 \cdot g(n)                         \\
                      \end{align*}
                     Also ist $ \prod^k_{i=1} f_i \in\Omega(g)$.








\end{enumerate}
\newpage

\subsection*{Aufgabe 5.3}
\begin{enumerate}
\item[a) Beh.:] Sei $C = 14$ und $n_0 = 1$. Dann ist $7n^6 + 3n^2 \in \Om(n^7)$.
\item[Bew.:]  Es gilt für $n\geq 1$:
   \begin{align*}
                                                       7n^6 + 3n^2          & \leq 7n^6 + 7n^2      \\
                                                                                       & =    7 ( n^6 + n^2)    \\
                                                                                       & \leq 7 ( n^6 + n^6)    \\
                                                                                       & = 14 n^6                     \\
                                                                                       & \leq 14 n^7    = C \cdot n^7                 \\
   \end{align*}
\item[b) Beh.:] Sei $C = 1$ und $n_0 = 1$. Dann ist $n^a \in \Om(e^n)$ für alle $a\in \N$.
\item[Bew.:]  Es gilt für $n\geq 1$:
   \begin{align*}
                                                       n^a                         & = a!\cdot\frac {n^{a}}{a!}       \\
                                                                                       & \leq \sum^{a!}_{k=0} \frac {n^{a}}{a!}      \\
                                                                                       & \leq \sum^{\infty}_{k=0} \frac {n^{k}}{k!}      \\
                                                                                       & = e^n   = C\cdot e^n                                                         \\
   \end{align*}
\item[c) Beh.:] Sei $C = 1$ und $n_0 = 3$. Dann ist $n!\in \Omega(2^n)$.
\item[Bew.:]  Es gilt für $n\geq 3$:
   \begin{align*}
                                                      n!                    \quad         & =                                        \quad \prod^n_{k=1} k    \\
                                                                                       & \stackrel{n \geq 3}{\geq} \quad  \prod^n_{k=1} 2   \\
                                                                                       & =                                        \quad 2^n  = C\cdot 2^n\\
   \end{align*}

\item[d) Beh.:] Sei $C = \frac 1 {\log_b(a)}$ und $n_0 = 1$. Dann ist $\log_a(n) \in \Theta( \log_b(n))$ für alle $a,b \in \R_{>1}$.
\item[Bew.:] Es gilt für $n\geq 1$:
   \begin{align*}
                                                    log_a(n)                      &  = \frac {\log_b(n)} {\log_b(a)}         \\
                                                                                       &  = \frac 1 {\log_b(a)}\cdot \log_b(n)  \\
                                                                                       & =  C \cdot \log_b(n)  \\
   \end{align*}
\end{enumerate}
\end{document}
