\documentclass[12pt]{article}

\usepackage{amsmath}
\usepackage{amssymb}    % noch mehr mathematische Symbole
\usepackage{german}
\usepackage[utf8]{inputenc}

\usepackage{geometry}                                                                    % Rand einstellen
\geometry{verbose,a4paper,tmargin=25mm,bmargin=25mm,lmargin=30mm,rmargin=35mm}


\usepackage{fancyhdr}
\pagestyle{fancy}
\lhead{Gruppe 2}
\rhead{Dimitrios-Deniz Antoulis / Timm-Oliver Stooß}
\cfoot{\thepage}
\renewcommand{\headrulewidth}{0.4pt}
\renewcommand{\footrulewidth}{0pt}


% Für natürliche Zahlen N, reelle Zahlen R und so weiter ...
\AtBeginDocument{%
  \let\mathbb\relax
  \DeclareMathAlphabet\PazoBB{U}{fplmbb}{m}{n}
  \newcommand{\mathbb}{\PazoBB}
}
\newcommand {\N}{\PazoBB{N}}   % natürliche Zahlen
\newcommand {\Z}{\PazoBB{Z}}   % ganze Zahlen
\newcommand {\Q}{\PazoBB{Q}}   % rationale Zahlen
\newcommand {\R}{\PazoBB{R}}   % reelle Zahlen


\newcommand {\Pm}{\mathcal{P}} % Potenzmenge
\newcommand {\Rm}{\mathcal{R}} % kalligravarphisches R

\begin{document}

\subsection*{ADS - Serie 4}
\subsection*{Aufgabe 4.2 Speicheraufwand Mergesort}
\begin{enumerate}
\item[Vor.:] 
  Sei $M : \N \to \N_0$ mit: 

\[
M(n) = 
\left\{
\begin{array}{lll}
&  0                                              & \quad :  \text{ falls } n = 1                \\ 
& n + M( \lceil n/2 \rceil )             & \quad :  \text{ sonst}     \\
\end{array}
\right.
\]




\item[Beh.:] Es gilt: \quad
$
  2(n-1) \quad \leq           \quad M(n) \quad \leq \quad       2 (n-1) + \lceil \log_2 n \rceil
$


\item[Bew Teil 1.:] Induktion über $l\in\N$. Erst   $2(n-1) \; \leq \; M(n)$
\item[IA.:] l = 1
\[
  2 (n-1) = 2 (1-1) = 2 \cdot 0 = 0 \quad \leq \quad M(1) = M(n)
\]



\item[IV.:]  Sei $l\in \N$ und es gelte für alle $n\in \{ 1, \dots l \}: \quad  2(n-1) \; \leq \; M(n)$



\item[IS.:] zz. Die Aussage gilt für $l+1$.\quad
                      $\big( \;2((l+1)-1) \quad \leq \quad M(l+1) \;\big)$

Nebenbeh.(*): $1 \leq \lceil (l+1)/2 \rceil \leq l$ \\
Bew.:
\[
1 \;\leq\; \lceil (1+1) /2 \rceil\; \leq\;  \lceil (l+1)/2 \rceil \;\leq\;   \lceil (l+l)/2 \rceil \;\leq \; \lceil 2l/2 \rceil\; \leq\; l
\]
Mit (*) darf die IV bei $\lceil (l+1)/2 \rceil$ angewendet werden.
\begin{align*}
       M(l+1)       =     &\;   (l+1) + M(\lceil (l+1)/2 \rceil )         & \text{ Def. von } M \\
                        \geq &\;   (l+1) + 2 (\lceil (l+1)/2 \rceil -1)     &  \text{ IV und (*) } \\
                        =      &\;   l+1 + 2 \lceil (l+1) / 2 \rceil  - 2       &    \\
                        =      &\;   l - 1 + \lceil (l+1) \rceil                    &   \\
                        =      &\;   l -1 + l +1                                        &    \\
                        =      &\;  2 l                                                     &     \\
                        =      &\; 2 (l +1 -1)                                         &    \\
                        =      &\;  2 ((l+1) -1)                                       &     \\
  \end{align*}

Also gilt $2(n-1) \quad \leq \quad M(n)$.

\item[Bew Teil 2.:] nun   $M(n) \;  \leq \;   2 (n-1) + \lceil \log_2 n \rceil$
\item[IA.:] l = 1
\[
  M(l) = M(1) = 0 \; \leq \; 0 + 0 =2 (1-1) + \lceil \log_2 1 \rceil
                                                         = 2 (l-1) + \lceil \log_2 l \rceil
\]



\pagebreak
\item[IV.:] Sei $l\in \N$ und es gelte für alle $n\in \{ 1, \dots l \}$: 
   $M(n) \; \leq \; 2 (n-1) + \lceil \log_2 n \rceil$.
\item[IS.:] zz.: $M(l+1) \quad \leq \quad 2 ((l+1)-1) + \lceil \log_2 (l+1) \rceil$.\\

\begin{align*}
      M(l+1)     =     &\;         (l+1) + M(\lceil (l+1) / 2 \rceil)                                                                    & \text{ Def. von } M \\
                      \leq &\;     (l+1) + 2 (\lceil(l+1) /2 \rceil -1) + \lceil \log_2 \lceil (l+1) / 2 \rceil \rceil         &  \text{ IV und (*) } \\
                      \leq &\;      l + 1 + \lceil l +1 \rceil - 2 +   \lceil \log_2 (l+1) \rceil - 1                                &  \text{ Hinweis}  \\
                      =    &\;       2 l -1 + \lceil \log_2 (l+1) \rceil &\\
                      \leq    &\; 2 l + \lceil \log_2 (l+1) \rceil &\\
                      =    &\; 2 ((l+1)-1) + \lceil \log_2 (l+1) \rceil &\\
  \end{align*}   
Also gilt $M(n) \;  \leq \;   2 (n-1) + \lceil \log_2 n \rceil$.





\end{enumerate}
\end{document}
