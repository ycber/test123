\documentclass[12pt]{article}

\usepackage{amsmath}
\usepackage{amssymb}    % noch mehr mathematische Symbole
\usepackage{german}
\usepackage[utf8]{inputenc}

\usepackage{fancyhdr}
\pagestyle{fancy}
\lhead{Gruppe 2}
\rhead{Dimitrios Dernis / Timm-Oliver Stooß}
\cfoot{\thepage}
\renewcommand{\headrulewidth}{0.4pt}
\renewcommand{\footrulewidth}{0pt}

\usepackage{geometry}                                                                    % Rand einstellen
\geometry{verbose,a4paper,tmargin=25mm,bmargin=25mm,lmargin=30mm,rmargin=35mm}

% Für natürliche Zahlen N, reelle Zahlen R und so weiter ...
\AtBeginDocument{%
  \let\mathbb\relax
  \DeclareMathAlphabet\PazoBB{U}{fplmbb}{m}{n}
  \newcommand{\mathbb}{\PazoBB}
}
\newcommand {\N}{\PazoBB{N}}   % natürliche Zahlen
\newcommand {\Z}{\PazoBB{Z}}   % ganze Zahlen
\newcommand {\Q}{\PazoBB{Q}}   % rationale Zahlen
\newcommand {\R}{\PazoBB{R}}   % reelle Zahlen


\newcommand {\Pm}{\mathcal{P}} % Potenzmenge
\newcommand {\Rm}{\mathcal{R}} % kalligravarphisches R

\begin{document}

\subsection*{ADS - Serie 4}
\subsection*{Aufgabe 4.3 Rechenaufwand Bubblesort}
\begin{enumerate}
\item[Vor.:] Programm siehe Aufgabenblatt
\item[Beh.:]  Bubblesort hat einen maximalen Rechenaufwand von $17n^2 - 32n +17$.
\item[Bew.:] Operationen berrechnen:\\
$\begin{array}{l|l}
 Zeile &  Operationen \\
\hline
1 & 0 \\
2 & 0 \\
3 & 4 \\
4 & 5 \\
5 & 4 \\
6 & 2 \\
7 & 4 \\
8 & 3\\
\end{array}$

Bei der inneren for-Schleife ergibt sich ein Rechenaufwand von:\\
3 für den Vergleich in der Schleife, 1 für i++ und dann Zeilen 5-8 aufaddiert insgesamt mal Anzahl Schleifendurchgänge + 1 für den letzten Vergleich + 1 für das Initialisieren i=0\\
= (3 + 1 + 4 + 2 + 4 + 4)  (n-i -1) +2 \\
= 17 (n-1) +2 \quad (i=0 ist der schlechteste Fall, der auftreten kann)\\
= 17 - 17 +2 \\
= 17n - 15 \\

Bei der äußeren Schleife ergibt sich ein Rechenaufwand von:\\
$(n-1) (17n-15) + 2\\
= 17n^2 - 15n - 17n - 15 - 2 \\
= 17n^2 - 32n + 17 \\$

Da es keine weiteren Anweisungen gibt, bleibt der Rechenaufwand kleiner gleich: $17n^2 - 32n + 17$ 
\end{enumerate}
\end{document}
